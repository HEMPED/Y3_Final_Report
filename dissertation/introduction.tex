\chapter{Introduction}\label{ch:introduction}

%Liquid democracy offers significant opportunities for enhancing the quality of collective decisions, particularly within systems that rely heavily on user-generated ratings. This project explores integrating liquid democracy into an existing rating platform to improve user engagement, decision-making accuracy, and the overall reliability of rating outcomes.
\section{Motivation - TODO (Finalise)}
Liquid democracy has strong theoretical appeal as a flexible and participatory decision-making model, but practical implementations remain rare and underexplored. Most existing research assumes idealised conditions, while real-world systems must contend with asynchronous participation, limited user engagement, and the potential for structural issues like delegation cycles or vote loss.

Vodle, as a platform for collective decision-making, presents an opportunity to explore how liquid democracy can be adapted to work in practice. This project is motivated by the need to understand and resolve the challenges of integrating delegation-based voting into an existing system, while improving the platform's fairness, expressiveness, and overall user experience.

\section{Vodle - TODO: finalise}
Vodle is an online platform where users participate in polls to vote on subjects through user created polls. Each poll contains a set of options, and users provide ratings for each option from 0 to 100, where a larger number means that they prefer the option more, using a slider. When the poll ends, the ratings submitted by voters are then aggregated and a result is calculated.

\textbf{can insert screenshots}

\section{Liquid Democracy}
Liquid democracy is a decision-making system that combines elements of both direct and representative democracy that offers a voter more flexibility than traditional voting models.

In direct democracy, every participant votes individually on each issue. This model offers the most individual input but can become impractical for large-scale decision-making due to the high level of participation required from each individual. As \cite{ford_delegative_2002} states, direct democracy assumes that all individuals are both willing and able to engage meaningfully with every decision, which is often not the case in large groups due to the variance in both the interest and knowledge of voters. The cognitive demand of staying informed on all matters, combined with the time commitment necessary for constant participation, makes direct democracy unmanageable at scale.

In a representative democracy, citizens elect officials who make decisions on their behalf for the duration of a fixed term. While this model is scalable and practical for large populations, it introduces several limitations. Elected representatives often make decisions based on party lines, personal convictions, or external influences such as lobbying groups, which may not accurately reflect the preferences of their constituents \citep{blum_liquid_2016}. In addition, because elections are infrequent, this system tends to be unresponsive to shifts in public opinion. Citizens are unable to easily adjust or retract their delegation, which limits their ability to influence decisions once representatives are in office \citep{blum_liquid_2016}. As a result, participation is both indirect and inflexible, which can lead to disengagement and dissatisfaction among voters.

Liquid democracy addresses these limitations by allowing voters to either cast their votes directly or delegate them to someone that they trust or to abstain from voting entirely \citep{blum_liquid_2016}. In comparison to a direct democracy, the bar for participation is lowered as voters no longer need to stay informed and engaged to pass a vote because they can trust a delegate to do it on their behalf.
These delegations can also be updated or revoked at any time, giving users more control over how their vote is used in comparison to a traditional representational democracy where your representative can only be changed at certain points in time.
% This enables users to rely on others' expertise while retaining the option to reclaim their vote at any point.

% The model promotes participation, flexibility, and informed decision-making, making it an appealing option for platforms seeking to improve their user experience and quality of voting result.
\section{Project Goal - TODO}
The project's main goal is to integrate liquid democracy into the vodle platform.

Key features include ranked delegation and vote splitting \dots

\section{Project Outline - TODO}
This report is structured as follows:
\textit{will add when report is written}