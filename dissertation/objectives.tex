\chapter{Project Objectives}
\label{ch:project_objectives}

This chapter defines the objectives of the project. Based on the challenges and needs identified during background research (Chapter~\ref{ch:background}), the objectives address both technical limitations in vodle's existing delegation system and theoretical concerns with liquid democracy models.

Objectives are divided into two categories:
\begin{itemize}
    \item \textbf{Core Objectives} -- Mandatory goals that are essential to delivering a functional, improved delegation system in vodle.
    \item \textbf{Extension Objective} -- A stretch goal intended to provide additional insights if time allows (see and Chapter~\ref{ch:project_management} and Chapter~\ref{ch:future} for details).
\end{itemize}

Each objective is broken down into specific functional and non-functional requirements to ensure measurability and verifiability. These requirements provide a structured framework for implementation and evaluation.

\section{Core Objectives (Mandatory)}
\begin{enumerate}
    \item \textbf{Implement a Core Delegation Model into Vodle:} Build a fully functional, cycle-safe delegation system, addressing the challenges of cycle prevention and transitive delegation (identified in Section~\ref{subsec:background_existing_delegation}).

    \item \textbf{Implement Ranked Delegation into Vodle:} Extend the system to allow users to specify multiple backup delegates, mitigating vote loss due to unavailable delegates (motivated by issues discussed in Section~\ref{subsec:background_ranked_delegation}).

    \item \textbf{Implement Weighted Delegation into Vodle:} Enable fractional delegation to multiple delegates using the trust matrix model (see Section~\ref{subsec:background_weighted_delegation}).

    \item \textbf{Implement Per-Option Delegation into Vodle:} Allow users to delegate different options to different delegates within the same poll, enhancing voter flexibility (inspired by Google Votes; see Section~\ref{subsec:google_votes}).
\end{enumerate}

\section{Extension Objective (Optional)}
\begin{enumerate}
    \item \textbf{Simulate Delegation Mechanisms:} Develop an agent-based modelling (ABM) simulation to evaluate the performance and robustness of different delegation mechanisms.
\end{enumerate}

\section{Project Requirements}
The following functional (F) and non-functional (NF) requirements operationalise the project objectives. Each requirement is stated to be specific, measurable, and testable, enabling systematic verification against the project goals.

\subsection{Requirements for Core Objective 1: Core Delegation Model}
\subsubsection{Functional Requirements}
\begin{itemize}
    \item \textbf{FR1.1:} The system shall allow users to invite others to act as delegates.
    \item \textbf{FR1.2:} Invited users shall be able to accept or decline delegation requests.
    \item \textbf{FR1.3:} Users shall be prevented from accepting their own delegation invitations.
    \item \textbf{FR1.4:} The system shall detect and prevent cyclic delegations.
    \item \textbf{FR1.5:} Users shall be able to view and revoke existing delegations.
    \item \textbf{FR1.6:} Delegations shall be resolved transitively to ensure accurate final vote attribution.
    \item \textbf{FR1.7:} Users shall be able to override their delegation by submitting direct votes.
\end{itemize}

\subsubsection{Non-Functional Requirements}
\begin{itemize}
    \item \textbf{NFR1.1:} Delegation data shall be stored in JSON format for compatibility with vodle's CouchDB backend.
    \item \textbf{NFR1.2:} Database schema changes must be backward compatible.
    \item \textbf{NFR1.3:} Users' voting and delegation choices shall remain private.
    \item \textbf{NFR1.4:} The delegation interface shall be intuitive and accessible.
\end{itemize}

\subsection{Requirements for Core Objective 2: Ranked Delegation}\label{req_ranked}
\subsubsection{Functional Requirements}
\begin{itemize}
    \item \textbf{FR2.1:} Users shall be able to specify up to three ranked delegates per poll.
    \item \textbf{FR2.2:} The MinSum rule shall be applied to resolve delegation paths.
    \item \textbf{FR2.3:} Users shall be able to override ranked delegations by direct voting.
    \item \textbf{FR2.4:} Users shall be able to modify or revoke ranked delegation preferences.
\end{itemize}

\subsubsection{Non-Functional Requirements}
\begin{itemize}
    \item \textbf{NFR2.1:} Ranked delegation data shall be stored in a JSON-compatible format.
    \item \textbf{NFR2.2:} The user interface for ranked delegation shall be clear and user-friendly.
\end{itemize}

\subsection{Requirements for Core Objective 3: Weighted Delegation}
\subsubsection{Functional Requirements}
\begin{itemize}
    \item \textbf{FR3.1:} Users shall be able to assign fractional votes to multiple delegates, ensuring the total weight does not exceed 0.99.
    \item \textbf{FR3.2:} Final vote weights shall be calculated using the trust matrix model.
\end{itemize}

\subsubsection{Non-Functional Requirements}
\begin{itemize}
    \item \textbf{NFR3.1:} Weighted delegation calculations shall occur client-side.
    \item \textbf{NFR3.2:} Data shall be stored as JSON to maintain CouchDB compatibility.
    \item \textbf{NFR3.3:} The weighted delegation UI shall provide intuitive controls (e.g., sliders) for assigning weights.
\end{itemize}

\subsection{Requirements for Core Objective 4: Per-Option Delegation}
\subsubsection{Functional Requirements}
\begin{itemize}
    \item \textbf{FR4.1:} Users shall be able to assign different delegates for different poll options.
    \item \textbf{FR4.2:} Delegation resolution shall be performed independently for each option.
    \item \textbf{FR4.3:} Users shall be able to override per-option delegations with direct votes.
    \item \textbf{FR4.4:} Users shall be able to view and manage per-option delegations easily.
\end{itemize}

\subsubsection{Non-Functional Requirements}
\begin{itemize}
    \item \textbf{NFR4.1:} Per-option delegation data shall be stored in JSON format.
    \item \textbf{NFR4.2:} The user interface shall clearly show assigned delegates per option.
\end{itemize}

\subsection{Requirements for Extension Objective: Simulate Delegation Mechanisms}
\subsubsection{Functional Requirements}
\begin{itemize}
    \item \textbf{FR5.1:} The simulation system shall model agents capable of voting, abstaining, or delegating.
    \item \textbf{FR5.2:} Simulations shall support core, ranked, and weighted delegation modes.
    \item \textbf{FR5.3:} Key parameters (number of agents, delegation probability, abstention rates) shall be configurable.
    \item \textbf{FR5.4:} Metrics such as super-voter concentration, delegation chain length, and vote loss shall be recorded.
    \item \textbf{FR5.5:} Simulation results shall be exportable in CSV or JSON format.
\end{itemize}

\subsubsection{Non-Functional Requirements}
\begin{itemize}
    \item \textbf{NFR5.1:} The simulation shall be lightweight and extensible.
    \item \textbf{NFR5.2:} Mesa shall be used as the agent-based modelling framework.
    \item \textbf{NFR5.3:} Simulations shall be reproducible with fixed random seeds.
\end{itemize}

