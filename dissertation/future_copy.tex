\chapter{Future Work}\label{ch:future}

Building on the foundations established during this project, several opportunities for enhancement have been identified. These include technical extensions to vodle's scalability and trust mechanisms, user interface improvements, and broader theoretical investigations through agent-based simulation of delegation dynamics.

\section{Agent-Based Modelling for Delegation Dynamics}

Agent-based modelling (ABM) provides a method for analysing how local decisions and interactions aggregate into system-level outcomes~\citep{bonabeau2002agent}. Although the simulation objective outlined in Section~\ref{sec:background_abm} was descoped during this project (see Section~\ref{ch:project_management}), ABM remains a promising approach for investigating delegation dynamics, particularly in large or complex settings.

Prior research has examined delegation behaviour primarily through observational studies and mathematical models. However, few studies have simulated how trust, strategic behaviour, and delegation reluctance evolve dynamically over time. This section outlines future directions for developing agent-based simulations that capture these behaviours, supported by appropriate evaluation metrics.

\subsection{Prior Simulation Studies of Liquid Democracy}

Several studies have explored delegation structures within liquid democracy systems, although most focus on static analyses rather than dynamic agent behaviour.

\subsubsection{Observational Studies}

\citet{kling2015votingbehaviourpoweronline} analysed delegation graphs from the LiquidFeedback platform, highlighting the emergence of super-voters. While informative, this study was observational, examining static snapshots rather than modelling how delegations evolve.

\subsubsection{Mathematical Modelling and Synthetic Evaluation}

\citet{brill_liquid_2022} evaluated various delegation rules on both synthetic and real-world datasets. However, their models assumed fixed agent preferences without modelling trust evolution or delegation reassignment.

\subsubsection{Formal Binary Models}

\citet{christoffBinaryVotingDelegable2017} introduced logical frameworks for binary voting settings, focusing on aggregation rules and rationality. Their work assumed a given delegation graph, without modelling agent decision-making processes.

\subsubsection{Summary}

While prior work has characterised the static properties of delegation networks, the dynamic formation and adaptation of delegation links based on agent-level behaviour remains underexplored. ABM provides a natural framework to investigate these dynamics.

\subsection{Baseline Agent-Based Model}

A baseline simulation could model agents defined by:

\begin{itemize}
    \item \textbf{Voting Intention}: An initial preference (\texttt{Yes}, \texttt{No}, or \texttt{Abstain}).
    \item \textbf{Delegation Willingness} ($w_i$): Propensity to delegate rather than vote directly.
    \item \textbf{Trust Vector} ($T_i$): Trust scores towards potential delegates.
    \item \textbf{Memory}: Record of past delegation outcomes for trust updates.
\end{itemize}

At each polling event:

\begin{enumerate}
    \item \textbf{Delegation Decision}: If trust and willingness exceed thresholds $\theta$ and $\theta'$, the agent delegates; otherwise, it votes directly.
    \item \textbf{Delegate Selection}: The agent selects the highest-trust available delegate.
    \item \textbf{Trust Update}: Trust increases if delegate outcomes align with the agent's preferences and decreases otherwise.
\end{enumerate}

Delegation chains would be resolved via a predefined rule (e.g., minimal rank sum), and the process would iterate across multiple polls.

\subsection{Extensions to the Baseline Model}

Realism can be enhanced through several extensions:

\subsubsection{Dynamic Trust Evolution}

Trust levels evolve based on delegate performance, with recent outcomes weighted more heavily to model realistic memory effects~\citep{casella_2022}.

\subsubsection{Delegation Reassignment}

Agents periodically re-evaluate their delegations, switching delegates or reverting to direct voting if trust falls below acceptable thresholds.

\subsubsection{User Behaviour and Strategic Delegation}

Incorporating varied agent behaviour, including:

\begin{itemize}
    \item \textbf{Overconfident Delegation}: Agents may choose to delegate even when could reduce the quality of the overall result~\citep{casella_2022}.
    \item \textbf{Reluctance to Delegate}: Agents could prefer self-representation despite potential benefits from delegation.
\end{itemize}

\subsection{Evaluation Metrics}

To assess the effectiveness and fairness of delegation mechanisms in agent-based simulations, a clear set of evaluation metrics is needed. These metrics should capture both the distribution of voting power and the structural properties of the delegation network.

\subsubsection{Voting Power Distribution}

Voting power in liquid democracy systems can be viewed as a transferable resource, similar to wealth or income. As it flows through delegation chains, it can become unevenly distributed, concentrating influence in the hands of a few. To quantify and monitor this distribution, standard inequality metrics from economics can be applied:

\begin{itemize}
    \item \textbf{Lorenz Curve}: The Lorenz curve plots the cumulative share of total voting power against the cumulative share of voters, sorted from least to most powerful~\citep{cowell_measuring_inequality}. A perfectly equal system would produce a diagonal line; greater curvature indicates greater inequality.
    \item \textbf{Gini Coefficient}: The Gini coefficient summarises the Lorenz curve into a single value between 0 and 1, where 0 represents perfect equality and 1 represents maximum inequality~\citep{cowell_measuring_inequality}.
    \item \textbf{Maximum Voting Weight}: Tracking the maximum voting weight held by any single agent provides a direct measure of super-voter emergence.
\end{itemize}

These measures, traditionally used in economic contexts to assess wealth inequality, are well-suited for evaluating how fairly voting power is distributed within a delegation network.

\subsubsection{Delegation Network Structure}

Analysing:

\begin{itemize}
    \item \textbf{Average Delegation Path Length}.
    \item \textbf{Cycle Frequency}: Even if prevented.
    \item \textbf{Connectivity and Robustness}.
\end{itemize}

\subsubsection{Comparative Analysis}

Comparing metrics across different delegation modes (core, ranked, weighted) to assess relative fairness and stability.

\subsection{Summary}

Agent-based simulations could reveal how delegation networks emerge, concentrate power, or adapt over time. This would inform design choices for more robust and equitable liquid democracy systems.

\section{Platform Extensions for Vodle}

While vodle's current delegation features substantially extend user flexibility, further enhancements could improve scalability, transparency, and long-term usability.

\subsection{Global Delegations}

Currently, delegations are poll-specific. Global delegations -- persisting across polls unless overridden -- would reduce friction for users who consistently trust certain delegates. This mirrors systems like LiquidFeedback~\citep{behrens_liquidfeedback_2014}.

Implementation would require:

\begin{itemize}
    \item Persistent delegation identifiers, scoped to delegation only (preserving poll privacy by preventing reverse-engineering of identifiers).
    \item Resolution logic combining global and local delegations.
    \item New UI components for managing global settings.
\end{itemize}

\subsection{Delegation Expiry Mechanisms}

Persistent delegations risk becoming outdated. Introducing optional expiry periods (e.g., six months) would encourage regular review and prevent passive concentration of voting power.

Expiry mechanisms would involve:

\begin{itemize}
    \item Expiry timestamps on delegations.
    \item Automatic fallback to backup delegates or direct voting.
    \item User notifications prior to expiry.
\end{itemize}

\subsection{Auditability of Delegation Chains}

Transparency into delegation paths increases trust. Vodle could offer auditability features inspired by the ``Golden Rule'' of Google Votes~\citep{hardt_google_2015}:

\begin{itemize}
    \item Users can view the number of delegation steps their vote traversed.
    \item Public delegates opt-in to visibility; others remain anonymised.
\end{itemize}

This balances accountability with privacy, strengthening user confidence.

\subsection{Database Scalability and Concurrency Issues}

Currently, vodle stores shared delegation data structures, such as the \\\texttt{direct\_delegation\_map}, within a single poll document. This design sufficed during initial development when delegation usage was limited. However, feedback from the project customer (see Section~\ref{sec:feedback}) highlighted significant risks associated with this approach as delegation features become more heavily used.

Since multiple users indirectly modify these shared maps when accepting, rejecting, or revoking delegations, concurrent updates risk overwriting each other -- a form of database collision. In CouchDB's document model, each write operation replaces the entire document. If two users attempt to update delegations simultaneously, one user's update can overwrite or erase the other's unless complex conflict resolution strategies are employed. This problem will intensify as user numbers grow, threatening the consistency and reliability of the delegation system.

Several potential strategies to address this issue were identified for future work:

\begin{itemize} \item \textbf{Per-User Delegation Documents}: Rather than storing all delegations in a single shared map, each user could maintain a private delegation document representing their outgoing delegations. The system would then dynamically resolve the global delegation graph client-side by aggregating relevant user documents during poll resolution. This would eliminate write contention between users, at the cost of increased client-side complexity and resolution time.

\item \textbf{Map Sharding}: An alternative is to partition the delegation maps by option, user group, or another shard key, storing smaller, independently modifiable documents. This reduces the likelihood of write conflicts but would require careful coordination when resolving delegations across shards.

\item \textbf{Delegation Transaction Log}: Instead of updating full maps, each delegation action (invite, accept, revoke) could be stored as an immutable event in a transaction log. Clients would replay the log to reconstruct the current delegation state. While this approach scales well and aligns with CouchDB's append-only model, it would require significant changes to vodle's frontend logic.

\item \textbf{Conflict-Aware Resolution}: If shared maps remain in use, conflict detection and manual resolution mechanisms could be implemented. Clients would detect CouchDB conflicts during sync and prompt users to retry operations. However, this approach may harm user experience and remains unresponsive under heavy concurrency.
\end{itemize}

\section{Summary}

This chapter has outlined potential directions for extending vodle's delegation system and for investigating the dynamics of liquid democracy through simulation. Agent-based modelling could offer insights into the evolution of trust and delegation behaviour, while platform-level enhancements such as global delegations, delegation expiry mechanisms, and privacy-preserving auditability would further strengthen usability and transparency.
Additionally, addressing database collision risks through per-user delegation documents or other strategies would ensure the system remains performant as user numbers grow. Together, these developments would support more flexible, scalable, and trustworthy decision-making systems based on liquid democracy principles.

