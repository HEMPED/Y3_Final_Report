\chapter{Project Management}\label{ch:project_management}

This chapter outlines the project's management approach, including the development methodology, planning, and reflections on the process. It also considers legal and ethical issues and assesses key risks associated with the project.

\section{Methodology}
The project followed an agile methodology, selected for its flexibility and its emphasis on iterative development and regular customer feedback. The work involved building a sequence of interdependent features into vodle; starting with a basic delegation mechanism and expanding to include ranked delegation, vote splitting, and per-option delegation. Since each feature built on the last, an iterative approach was ideal to ensure compatibility and to allow design decisions to adapt over time.

Agile was particularly well suited to this project due to the presence of an active ``customer'' figure: Jobst Heitzig, the co-supervisor and original creator of vodle. Heitzig provided clarified system expectations and helped shape design decisions based on the real-world use case. Fortnightly meetings were held with both Jobst Heitzig and Markus Brill, to review progress and incorporate their feedback into the next development cycle. This tight feedback loop is a core principle of agile, allowing the project to stay aligned with user needs and system goals.

Alternative models such as Waterfall were considered but ultimately rejected due to their inflexibility. Waterfall could have provided clear documentation at each phase and well-defined milestones, which initially seemed beneficial for implementing interconnected voting delegation features. However, Waterfall requires defining the full scope of the project upfront and offers limited room for revision - something that would have been impractical due to the project's shorter timeframe and the evolving nature of requirements as features were tested.

Elements of Scrum were also considered, as it is one of the most widely adopted agile frameworks (63\% of agile teams use Scrum \citep{versionone2020stateofagile}). The sprint-based development cycles, clearly defined roles, and structured ceremonies like sprint planning and reviews were attractive features that could have facilitated focused implementation and effective communication. However, daily stand-up meetings and fixed-length sprints were not feasible for this project, as all parties involved (myself and the two supervisors) had other commitments. The small team size didn't warrant all Scrum ceremonies, and the academic nature of the project called for more flexible review cycles. Instead, progress was reviewed every two weeks, ensuring feedback could still be gathered and acted upon without adding unnecessary scheduling pressure.

\textit{move to start?}
Each development cycle produced a working, testable feature that could be evaluated and integrated into the overall system. This approach reduced the risk of late-stage errors and helped maintain steady progress throughout the project. Agile's iterative and feedback-driven structure was a natural fit for the technical and collaborative demands of this work.
\section{Plan}

The project plan was organised into objectives (see section \ref{ch:project_objectives}) that built on one another in sequence:

\begin{itemize}
    \item \textbf{Core Objective 1:} Implement a Core Delegation Model into Vodle.
    \item \textbf{Core Objective 2:} Implement Ranked Delegation into Vodle.
    \item \textbf{Core Objective 3:} Implement a Vote Splitting Delegation Mechanism into Vodle.
    \item \textbf{Core Objective 4:} Implement the Ability to Delegate Individual Options to Different Users.
    \item \textbf{Extension Objective 1:} Simulate Delegation Mechanisms.
\end{itemize}

This objective-led structure was well-suited to the agile approach, allowing each milestone to be treated as an iteration with a deliverable at the end. A Gantt chart (see below) was created to visualise the project timeline and to track dependencies and progress.

\textbf{TODO: Gantt Chart}
\section{Risk Assessment - TODO}
\begin{table}[H]
\centering
\begin{tabular}{|p{4.5cm}|p{2cm}|p{8cm}|}
\hline
\textbf{Risk} & \textbf{Likelihood} & \textbf{Mitigation Strategy} \\
\hline
Breaking the live vodle site during development & Medium & Use Git branching to isolate development from production environments. Conduct local testing before deployment. \\
\hline
Feature complexity exceeds estimates & High & Prioritise core objectives and maintain flexibility in scope. \\
\hline
Lack of engagement from supervisors or stakeholders & Low & Maintain regular communication through scheduled meetings. \\
\hline
Data loss or corruption & Low & Use Git for version control and take regular local backups. \\
\hline
\end{tabular}
\caption{Key risks identified and their mitigation strategies}\label{tab:risk-assessment}
\end{table}

The following sections provide a detailed analysis of the risks identified in the risk assessment table (see Table~\ref{tab:risk-assessment}). Each risk is discussed individually, outlining its implications and the strategies proposed to mitigate potential issues.

\subsection*{Breaking the Live Vodle Site During Development}

\textbf{Likelihood:} Medium

\textbf{Description:} Modifications to the existing vodle platform could unintentionally introduce downtime or impair existing functionality on the live website. Any disruptions could negatively impact real users' interactions, leading to dissatisfaction and loss of trust in the platform.

\textbf{Mitigation:} Development activities will utilise Git branching to isolate new code from the production environment. Features will be developed and rigorously tested in local or staging environments before integration with the live deployment. Incremental rollouts and thorough pre-deployment testing will further help identify potential problems early, allowing quick remediation or rollback.

\subsection*{Feature Complexity Exceeds Estimates}

\textbf{Likelihood:} High

\textbf{Description:} Advanced features such as ranked delegation and vote splitting may prove more complex than initially anticipated. Unexpected complexity can lead to delays, reduced functionality, or incomplete implementations, potentially affecting the project's timeline and deliverables.

\textbf{Mitigation:} Core objectives have been clearly defined and prioritised, ensuring focus remains on essential functionality. In cases of higher than anticipated complexity, resources will be redirected towards completing critical core features first, while the extension objective (agent based modelling) can be scaled back or postponed as needed. Regular agile reviews will monitor progress closely, facilitating early identification and management of complexity-related issues.

\subsection*{Lack of Engagement from Supervisors or Stakeholders}

\textbf{Likelihood:} Low

\textbf{Description:} Regular feedback and engagement from supervisors and stakeholders are crucial to ensure alignment with project goals, requirements, and user expectations. Insufficient feedback could result in misaligned implementations or objectives that do not fully meet user needs.

\textbf{Mitigation:} Fortnightly meetings have been scheduled with both the primary supervisor (Markus Brill) and the co-supervisor (Jobst Heitzig), who also fulfils the role of the project customer. This structured schedule ensures consistent opportunities for input and feedback. Additionally, a Telegram group chat is available to handle urgent queries and maintain ongoing dialogue.

\subsection*{Data Loss or Corruption -- Code}

\textbf{Likelihood:} Low

\textbf{Description:} Development activities pose a risk of code loss or corruption due to accidental deletion, unintended changes, or version conflicts. Such incidents could significantly delay development and necessitate additional time for recovery.

\textbf{Mitigation:} Version control will be rigorously maintained using Git, with frequent commits and descriptive commit messages ensuring traceability. Regular backups of the repository will be taken to safeguard against accidental loss, providing straightforward recovery paths when needed.

\subsection*{Data Loss or Corruption -- Database}

\textbf{Likelihood:} Low

\textbf{Description:} While unlikely, corruption of the CouchDB database during development could occur due to improper schema modifications or accidental changes.
%Given the development nature of the database, such corruption primarily affects development efficiency rather than user data or long-term stability.

\textbf{Mitigation:} No mitigation --- in the event of data corruption, the development database will be reset to its original state. As all data in the database is poll and user specific, it does not impact development as no important data is stored in the production environment.

\section{Risk Management Reflection - TODO}

\section{Legal and Ethical Considerations}

As vodle may eventually be used to gather votes on sensitive topics, care was taken to ensure privacy and fairness. Delegation chains are resolved internally and never publicly exposed, preserving the confidentiality of voter relationships. No personal data was collected or processed for the purposes of this project, so no changes to the platform's terms of service were required.

\section{Overall/Self Reflection - TODO}
