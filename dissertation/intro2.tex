\chapter{Introduction}\label{ch:introduction}

\section{Context and Motivation}

Collective decision-making is a cornerstone of modern governance, but traditional approaches often struggle to balance transparency, efficiency, and scalability. \textit{Direct democracy} grants citizens full participation in decisions, promoting transparency and individual control; however, it does not scale easily to large or complex societies. In contrast, \textit{representative democracy} improves scalability by delegating authority to elected officials, but at the cost of reduced personal influence and flexibility.

Liquid democracy emerges as a hybrid model that balances these trade-offs by allowing individuals either to vote directly or to delegate their voting power to trusted peers, combining the transparency of direct democracy with the scalability of representation \citep{ford_delegative_2002, blum_liquid_2016}. Unlike traditional representation, where authority is typically fixed for a term, liquid democracy supports dynamic, reversible delegations, enabling voters to adapt their participation to changing circumstances, expertise, or trust.

This project applies liquid democracy to \textit{vodle}, a web-based platform for participatory decision-making, where users rate options on a scale (from 1 to 100) using the MaxParC rating system. While vodle enables users to express nuanced preferences across decision options, it shares a common challenge inherent to direct democracy: as decisions become more numerous and complex, many users lack the time, expertise, or motivation to engage with every vote \citep{ford_delegative_2002, blum_liquid_2016}. This disengagement can lead to underrepresentation and diminished decision quality. Without delegation, vodle risked low participation on complex decisions, undermining its goal of fostering broad consensus. This motivated the integration of liquid democracy mechanisms to keep users involved.

By integrating liquid democracy mechanisms, this project aims to address these limitations by providing a flexible way for users to remain active participants even when direct engagement is impractical, helping to improve inclusiveness, scalability, and responsiveness while preserving transparency and individual control.

\section{Project Goals}

The core goal of this project is to design and implement a flexible, scalable, and transparent liquid democracy system for vodle. The system should allow users to delegate their votes in ways that preserve autonomy, promote engagement, and accommodate a range of different behaviours and participation levels.

Objectives include:
\begin{itemize}
    \item Allowing users to delegate their voting power to others while ensuring that all votes are correctly attributed and included in the final outcome.
    \item Supporting ranked delegation with backup options.
    \item Supporting weighted delegation based on trust.
    \item Allowing delegation on a per-option basis.
\end{itemize}

\section{Structure of This Report}

This report is organised to guide the reader from theoretical foundations through to practical implementation, evaluation, and reflection. Each chapter builds logically upon the previous, tracing the project's development from research to outcomes:

\begin{itemize}
    \item Chapter~\ref{ch:background} reviews the theoretical background of liquid democracy, identifying key challenges such as delegation cycles, abstentions, and super-voters. It further analyses real-world implementations, including LiquidFeedback and Google Votes, to draw lessons from existing systems. Finally, it examines vodle's technical architecture and partially implemented delegation system, establishing the practical constraints and requirements for the project.

    \item Chapter~\ref{ch:project_objectives} defines the project's objectives, translating the issues identified in Chapter~\ref{ch:background} into specific functional and non-functional requirements. These objectives provide a clear framework to guide implementation and evaluation.

    \item Chapter~\ref{ch:design_implementation} details the design and implementation of the delegation mechanisms within vodle. Each delegation model---core delegation, ranked delegation, weighted delegation, and per-option delegation---is discussed, with explanations of key technical decisions and how they address the project objectives.

    \item Chapter~\ref{ch:evaluation} evaluates the system through unit testing, performance analysis, and a critical assessment of how successfully the final implementation meets the defined objectives.

    \item Chapter~\ref{ch:project_management} reflects on project management aspects, including the development methodology, planning and scheduling, risk assessment, and ethical considerations, and how these processes shaped the technical outcomes.

    \item Chapter~\ref{ch:future} proposes directions for future work, suggesting potential enhancements to delegation features and the use of agent-based modelling to explore delegation dynamics at a larger scale.

    \item Chapter~\ref{ch:conclusions} concludes by summarising the project's contributions and reflecting on its achievements.
\end{itemize}

\section{Contributions}

In summary, this project contributes:
\begin{itemize}
    \item An enhanced liquid democracy module for vodle addressing cycle detection, vote loss, and super-voter issues.
    \item New delegation modes (ranked, weighted, per-option) implemented with client-side logic optimised for CouchDB storage.
    \item An evaluation demonstrating scalability and robustness of the proposed mechanisms through performance testing.
\end{itemize}
