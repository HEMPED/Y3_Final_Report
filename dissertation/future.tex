\chapter{Future Work}
This project established a foundation for flexible, transitive, ranked, and weighted delegation within vodle. While the primary development goals were met, there are several areas where the system could be further extended or refined. These include the development of agent-based simulation tools to better understand delegation dynamics, and the expansion of vodle's delegation features to support a wider range of use cases. Addressing these areas would strengthen both the theoretical understanding and practical deployment of liquid democracy mechanisms within platforms like vodle.

\section{Agent-Based Modelling for Delegation Dynamics}

Agent-based modelling (ABM) provides a methodology for analysing how local decisions and interactions aggregate into system-level outcomes~\citep{bonabeau2002agent}. Although the simulation objective outlined in Section~\ref{sec:background_abm} was ultimately descoped during this project (see Section~\ref{ch:project_management}), agent-based modelling remains a promising approach for investigating the dynamics of delegation within liquid democracy systems, particularly in large or complex decision-making contexts.

Prior research has explored delegation behaviour primarily through observational studies or mathematical analysis. However, few studies have examined how trust, strategic behaviour, or delegation reluctance evolve over time. This section outlines directions for developing agent-based simulations that capture these dynamic behaviours, supported by suitable evaluation metrics.

\subsection{Prior Simulation Studies of Liquid Democracy}

Several previous studies have examined the delegation structures that emerge within liquid democracy systems, although most have focused either on observational analysis of real-world systems or on mathematical modelling of delegation resolution processes, rather than on simulating dynamic agent behaviours.

\subsubsection{Observational Studies}

\citet{kling2015votingbehaviourpoweronline} analysed delegation graphs from the LiquidFeedback platform, using real-world data from the German Pirate Party. Their work focused on the structure of the delegation network, identifying the emergence of ``super-voters''. Although highly informative, this study was observational in nature, examining static snapshots of delegation graphs rather than simulating how individual users might form or revise their delegation choices over time.

% \subsubsection{Mathematical Modelling and Synthetic Evaluation}

% \citet{brill_liquid_2021} formalised a framework for ranked delegations and proposed several delegation rules for resolving delegation paths. Their analysis combined theoretical results with experimental evaluation on both synthetic and real-world datasets. However, their data generation methods created fixed agent preferences and delegation options; dynamic behavioural evolution, trust updates, or delegation reassignment over time were not modelled. As such, while their work offers critical insights into the properties of delegation rules under different conditions, it does not address dynamic aspects of delegation behaviour.

\subsubsection{Mathematical Modelling and Synthetic Evaluation}

\citet{brill_liquid_2021} formalised a framework for ranked delegations and proposed several delegation rules for resolving delegation paths. Their analysis involved theoretical formalisation of delegation rules and experimental evaluation on both synthetic and real-world datasets (such as partial networks from Facebook). However, their data generation methods created fixed agent preferences and delegation options; dynamic behavioural evolution, trust updates, or delegation reassignment over time were not modelled. As such, while their work offers critical insights into the properties of delegation rules under different conditions, it does not address dynamic aspects of delegation behaviour.

\subsubsection{Formal Binary Models}

\citet{christoffBinaryVotingDelegable2017} introduced a logical framework for analysing liquid democracy in binary voting settings, where voters must decide on yes/no questions. Their analysis focused on collective rationality and the presence of delegation cycles, examining how different aggregation rules could resolve delegation paths and ensure consistency. Like the other approaches, their framework assumed a given delegation graph and did not model the agent-level processes by which delegation links are formed.

\subsubsection{Summary}

Although prior work has provided valuable insights into the static properties of delegation networks and the mathematical characteristics of delegation resolution, the dynamic formation and evolution of delegation networks based on individual agent behaviour remains underexplored. Agent-based modelling offers a natural framework to investigate these phenomena.

\subsection{Baseline Agent-Based Model for Delegation Dynamics}

A baseline agent-based model could simulate the fundamental processes by which agents decide whether to vote directly or delegate. Each agent would be defined by the following attributes:

\begin{itemize}
    \item \textbf{Voting Intention}: An initial preference (\texttt{Yes}, \texttt{No}, or \texttt{Abstain}).
    \item \textbf{Delegation Willingness} ($w_i$): A score in $[0,1]$ representing the agent's propensity to delegate.
    \item \textbf{Trust Vector} ($T_i$): Trust levels towards potential delegates.
    \item \textbf{Memory}: A record of past delegation outcomes to inform trust updates.
\end{itemize}

In each polling event, agents would follow a decision-making process:

\begin{enumerate}
    \item \textbf{Delegation Decision}: If the agent's trust in others exceeds a delegation threshold $\theta$ and their willingness $w_i$ exceeds a secondary threshold $\theta'$, the agent delegates; otherwise, the agent votes directly.
    \item \textbf{Delegate Selection}: The agent selects the individual with the highest trust score above the threshold.
    \item \textbf{Trust Update}: After the poll outcome, trust is reinforced if the delegate's action aligns with the agent's preference, and decayed otherwise, bounded within $[0,1]$.
\end{enumerate}

Delegation chains would be resolved according to a predefined rule (e.g., minimal rank sum), and the process would repeat across multiple polls, allowing for the study of how delegation structures evolve over time.

\subsection{Extensions to the Baseline Model}

Several extensions could enhance the baseline model's realism and explanatory power:

\subsubsection{Dynamic Trust Evolution}

Agents could dynamically adjust their trust levels based on delegate performance. Trust reinforcement would occur when a delegate's actions align with the agent's preferences, while trust decay would occur otherwise. Incorporating memory-weighted trust updates, where recent experiences have greater influence, would further reflect realistic human decision-making patterns.

\subsubsection{Delegation Reassignment}

Agents whose trust in a delegate drops below a threshold could seek alternative delegates or revert to direct voting. This reassignment mechanism would model the fluid nature of trust relationships and delegation decisions.

\subsubsection{User Behaviour and Strategic Delegation}

Simulations could incorporate varied behavioural profiles, including:

\begin{itemize}
    \item \textbf{Overconfident Delegation}: As observed by \citet{casella_2022}, voters may delegate even when it reduces decision quality.
    \item \textbf{Reluctance to Delegate}: Some voters may resist delegating even when beneficial, due to distrust or control preferences.
\end{itemize}

Exploring these behaviours would allow simulations to assess system reliability under more realistic agent populations.

\subsection{Evaluation Metrics for Simulation Outcomes}

To assess the effectiveness and fairness of delegation mechanisms in agent-based simulations, a clear set of evaluation metrics is needed. These metrics should capture both the distribution of voting power and the structural properties of the delegation network.

\subsubsection{Voting Power Distribution}

Voting power in liquid democracy systems can be viewed as a transferable resource, similar to wealth or income. As it flows through delegation chains, it can become unevenly distributed, concentrating influence in the hands of a few. To quantify and monitor this distribution, standard inequality metrics from economics can be applied:

\begin{itemize}
    \item \textbf{Lorenz Curve}: The Lorenz curve plots the cumulative share of total voting power against the cumulative share of voters, sorted from least to most powerful~\citep{cowell_measuring_inequality}. A perfectly equal system would produce a diagonal line; greater curvature indicates greater inequality.
    \item \textbf{Gini Coefficient}: The Gini coefficient summarises the Lorenz curve into a single value between 0 and 1, where 0 represents perfect equality and 1 represents maximum inequality~\citep{cowell_measuring_inequality}.
    \item \textbf{Maximum Voting Weight}: Tracking the maximum voting weight held by any single agent provides a direct measure of super-voter emergence.
\end{itemize}

These measures, traditionally used in economic contexts to assess wealth inequality, are well-suited for evaluating how fairly voting power is distributed within a delegation network.

\subsubsection{Delegation Network Structure}

The structure of the delegation network itself would also provide valuable insights:

\begin{itemize}
    \item \textbf{Average Delegation Path Length}: Indicates the typical number of delegation steps from voters to their representatives.
    \item \textbf{Cycle Frequency}: Measures the frequency of attempted cycles, even if prevented by the system.
    \item \textbf{Network Connectivity}: Assesses whether the delegation network remains cohesive or fragments into isolated groups.
\end{itemize}

\subsubsection{Comparative Analysis}

By applying these metrics across different delegation models -- such as ranked delegation (see Section~\ref{sec:design_ranked_delegation}) or weighted delegation (see Section~\ref{sec:design_per_option_delegation}) -- simulations could systematically compare the performance of different mechanisms. Longitudinal analysis could further reveal whether delegation networks tend toward stability or increasing inequality over time.

\subsection{Summary}

Agent-based modelling presents a promising approach for exploring the dynamic behaviours underlying delegation systems. By modelling individual agent decisions and trust evolution, future simulations could provide deeper insights into the stability, fairness, and scalability of liquid democracy mechanisms.


\section{Potential Extensions for Vodle}

While the implementation described in this project significantly improves delegation flexibility and expressiveness in vodle, several further extensions could enhance the platform's scalability, usability, and transparency. This section outlines selected directions for future development.

\subsection{Global Delegations}

In the current implementation, delegations are issued individually for each poll, requiring users to manually re-invite their preferred delegates whenever they join a new poll. For users who consistently wish to delegate their participation, this process introduces unnecessary overhead.

Global delegations would allow users to establish persistent delegations that automatically apply across multiple polls until explicitly overridden or revoked. This design mirrors the structure used in LiquidFeedback~\citep{behrens_liquidfeedback_2014}, where users can define delegations globally, per subject area, or per issue, with more specific delegations taking precedence.

Implementing global delegations in vodle would require changes to the platform's identity management system. At present, poll-specific user identifiers are used to preserve privacy and prevent cross-poll tracking. Supporting global delegations would necessitate introducing persistent identifiers scoped exclusively to the delegation system, ensuring delegation persistence without compromising poll privacy.

Delegations would be stored within user profile metadata, and the resolution process would prioritise per-poll delegations before falling back to any global delegation. Cycle prevention mechanisms would also need to account for both global and local delegation links during validation.

At the user interface level, additional screens would allow users to manage global delegations, review their active settings, and revoke delegations as needed. Clear indicators during voting would distinguish between global and poll-specific delegations, maintaining transparency about how votes are cast.

Introducing global delegations would reduce management friction for active users, accommodate varying engagement levels, and further scale vodle's implementation of liquid democracy.

\subsection{Delegation Expiry Mechanisms}

Although persistent delegations improve convenience, they risk becoming outdated if users' trust relationships change over time. To mitigate this, vodle could introduce optional expiry periods for delegations.

When issuing a delegation, users would have the ability to specify an expiry period, such as six months or one year. Once expired, the delegation would automatically become inactive, prompting the user to review, renew, or modify their delegation choice.

Delegation expiry would encourage users to regularly reconsider their trust relationships, prevent passive accumulation of voting power, and promote a dynamic and up-to-date delegation network. Technically, this extension would involve associating an optional expiry timestamp with each delegation and modifying the resolution process to treat expired delegations as inactive, defaulting either to backup delegates or to direct voting.

Providing reasonable default expiry options, combined with user notifications prior to expiry, would help balance user convenience with network integrity.

\subsection{Auditability of Delegation Chains}

Transparency in how votes are cast and propagated is crucial for building trust in liquid democracy systems. One potential extension is to provide users with greater auditability of their delegation chains: showing how their vote travels through the network to reach its final casting point.

However, auditability must be balanced carefully against delegate privacy. In vodle, delegations and votes are private, and no user identity should be exposed unless explicitly permitted. Drawing inspiration from the ``Golden Rule of Liquid Democracy'' proposed in Google Votes~\citep{hardt_google_2015}, a privacy-preserving auditability model can be adopted:  
users can trace how their own votes were used, but identities of intermediaries are only shown if those users have explicitly opted into public visibility.

Practically, vodle could implement a system where:
\begin{itemize}
    \item Users see the number of steps their delegated vote has traversed.
    \item If an intermediary has opted into being a public delegate, their pseudonym is displayed.
    \item Otherwise, intermediate steps are anonymised but structurally visible.
\end{itemize}

This approach preserves voter transparency without compromising delegate privacy, and aligns with best practices in privacy-respecting liquid democracy systems.

\section{Summary}

This chapter has outlined potential directions for extending vodle's delegation system and for investigating the dynamics of liquid democracy through simulation. Agent-based modelling could offer insights into the evolution of trust and delegation behaviour, while platform-level enhancements such as global delegations, delegation expiry mechanisms, and privacy-preserving auditability would further strengthen usability and transparency. Together, these developments would support more flexible, scalable, and trustworthy decision-making systems based on liquid democracy principles.
