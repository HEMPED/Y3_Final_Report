\chapter{Introduction}\label{ch:introduction}

\section{Context and Motivation}

Collective decision-making is essential across a wide range of settings, from online communities to formal organisations. However, traditional models face challenges as participation scales. In direct democracy, individuals vote on every issue, promoting transparency and personal agency, but engagement tends to decline as decision-making burdens increase \citep{ford_delegative_2002}. Representative democracy addresses scalability by electing intermediaries, but often does so at the cost of accountability and flexibility \citep{blum_liquid_2016}.

Liquid democracy emerges as a hybrid approach designed to balance these trade-offs. Individuals can either vote directly or delegate their vote to a trusted peer, blending the transparency of direct democracy with the scalability of representation \citep{ford_delegative_2002, blum_liquid_2016}. Unlike traditional representation, where elected representatives remain fixed for a term, liquid democracy allows delegations to change at any time and enables individuals to reclaim their voting power whenever they choose. This flexibility helps systems adapt to varying levels of user engagement and expertise.

This project applies liquid democracy to \textit{vodle}, an existing web-based platform for group decision-making. While vodle supports nuanced expression of preferences across decision options, it inherits a common weakness: users may lack the time, expertise, or motivation to engage with every decision, leading to underrepresentation and disengagement \citep{ford_delegative_2002, blum_liquid_2016}.

Integrating liquid democracy features aims to address these limitations by giving users more flexible ways to participate. Through delegation, users can remain active in decision processes even when unable or unwilling to vote directly, enhancing the platform's inclusiveness, scalability, and responsiveness -- while preserving transparency and individual control.

\section{Project Goals}

The goal of this project is to design and implement a flexible, scalable, and transparent liquid democracy system for vodle. The system should allow users to delegate their votes in ways that preserve autonomy, promote engagement, and accommodate a range of different behaviours and participation levels.

Specific objectives include:
\begin{itemize}
    \item Allowing users to delegate their voting power to others while ensuring that all votes are correctly attributed and included in the final outcome.
    \item Supporting ranked delegation with backup options.
    \item Supporting weighted delegation based on trust.
    \item Allowing delegation on a per-option basis.
    \item Ensuring that users can revoke delegations or override them by voting directly at any time.
\end{itemize}

\section{Structure of This Report}

This report is organised as follows:

\begin{itemize}
    \item Chapter~\ref{ch:background} reviews background material relevant to the project, including the principles of liquid democracy, challenges such as delegation cycles and abstentions, variations like ranked and weighted delegation, real-world implementations, the use of agent-based modelling, and the technical context of vodle.
    \item Chapter~\ref{ch:project_objectives} defines the objectives and requirements of the project, establishing the functionality and quality expectations for the liquid democracy features.
    \item Chapter~\ref{ch:design_implementation} presents the design and implementation of the system, describing the delegation mechanisms developed and the technical solutions employed to meet the project requirements.
    \item Chapter~\ref{ch:evaluation} evaluates the system through unit testing, performance analysis, and an assessment of how well the project objectives have been achieved.
    \item Chapter~\ref{ch:project_management} discusses the project management approach, including the methodology used, the project plan and changes, risk assessment, and ethical considerations.
    \item Chapter~\ref{ch:future} proposes directions for future work, including extensions to vodle and simulation studies of delegation dynamics.
    \item Chapter~\ref{ch:conclusions} summarises the project's outcomes and provides concluding reflections.
\end{itemize}
