\chapter{Project Management}

This chapter outlines the project's management approach, including the development methodology, planning, and reflections on the process. It also considers legal and ethical issues and assesses key risks associated with the project.

\section{Methodology}
The project followed an agile methodology, selected for its flexibility and its emphasis on iterative development and regular customer feedback. The work involved building a sequence of interdependent features into vodle; starting with a basic delegation mechanism and expanding to include ranked delegation, vote splitting, and per-option delegation. Since each feature built on the last, an iterative approach was necessary to ensure compatibility and to allow design decisions to adapt over time.

Agile was particularly well suited to this project due to the presence of an active ``customer'' figure: Jobst Heitzig, the co-supervisor and original creator of vodle. His role went beyond academic supervision - he provided clarified system expectations and helped shape design decisions based on the real-world use case. Fortnightly meetings were held with both Jobst and the academic supervisor, Markus Brill, to review progress and incorporate their feedback into the next development cycle. This tight feedback loop is a core principle of agile, allowing the project to stay aligned with user needs and system goals.

Alternative models such as Waterfall were considered but ultimately rejected due to their inflexibility. Waterfall could have provided clear documentation at each phase and well-defined milestones, which initially seemed beneficial for implementing interconnected voting delegation features. However, Waterfall requires defining the full scope of the project upfront and offers limited room for revision - something that would have been impractical due to the project's shorter timeframe and the evolving nature of requirements as features were tested. Agile allowed the system to evolve in parallel with the design and testing of new ideas.

Elements of Scrum were also considered, as it is one of the most widely adopted agile frameworks, used by 63\% of teams use Scrum \citep{versionone2020stateofagile}. The sprint-based development cycles, clearly defined roles, and structured ceremonies like sprint planning and reviews were attractive features that could have facilitated focused implementation and effective communication. However, daily stand-up meetings and fixed-length sprints were not feasible for this project, as all parties involved (myself and the two supervisors) had other commitments. The small team size didn't warrant all Scrum ceremonies, and the academic nature of the project called for more flexible review cycles. Instead, progress was reviewed every two weeks, ensuring feedback could still be gathered and acted upon without adding unnecessary scheduling pressure.

Each development cycle produced a working, testable feature that could be evaluated and integrated into the overall system. This approach reduced the risk of late-stage errors and helped maintain steady progress throughout the project. Agile's iterative and feedback-driven structure was a natural fit for the technical and collaborative demands of this work.
\section{Plan}

The project plan was organised into objectives (see section \ref{section:project_objectives}) that built on one another in sequence:

\begin{itemize}
    \item \textbf{Core Objective 1:} Implement a Core Delegation Model into Vodle.
    \item \textbf{Core Objective 2:} Implement Ranked Delegation into Vodle.
    \item \textbf{Core Objective 3:} Implement a Vote Splitting Delegation Mechanism into Vodle.
    \item \textbf{Core Objective 4:} Implement the Ability to Delegate Individual Options to Different Users.
    \item \textbf{Extension Objective 1:} Simulate Delegation Mechanisms.
\end{itemize}

This objective-led structure was well-suited to the agile approach, allowing each milestone to be treated as an iteration with a deliverable at the end. A Gantt chart (see below) was created to visualise the project timeline and to track dependencies and progress.

\textbf{TODO: Gantt Chart}
\section{Risk Assessment - TODO}
\begin{table}[h!]
\centering
\begin{tabular}{|p{4.5cm}|p{2cm}|p{8cm}|}
\hline
\textbf{Risk} & \textbf{Likelihood} & \textbf{Mitigation Strategy} \\
\hline
Breaking the live vodle site during development & Medium & Use Git branching to isolate development from production environments. Conduct local testing before deployment. \\
\hline
Feature complexity exceeds estimates & High & Prioritise core objectives and maintain flexibility in scope. \\
\hline
Lack of engagement from supervisors or stakeholders & Low & Maintain regular communication through scheduled meetings. \\
\hline
Data loss or corruption & Low & Use Git for version control and take regular local backups. \\
\hline
\end{tabular}
\caption{Key risks identified and mitigation strategies}
\label{tab:risk-assessment}
\end{table}

\section{Legal and Ethical Considerations}

As vodle may eventually be used to gather votes on sensitive topics, care was taken to ensure privacy and fairness. Delegation chains are resolved internally and never publicly exposed, preserving the confidentiality of voter relationships. No personal data was collected or processed for the purposes of this project, so no changes to the platform's terms of service were required.

\section{Risk Management Reflection - TODO}

\section{Overall/Self Reflection - TODO}

% From a personal standpoint, the project management approach allowed consistent progress without becoming overly rigid. The combination of a clear objective structure and regular supervisor input helped maintain focus and motivation. While some features proved more technically involved than initially expected, the flexibility of the agile methodology ensured that deliverables were still met.

% Overall, agile development supported the implementation of liquid democracy features in a manageable and well-organised way, resulting in a system that is both functional and extensible within vodle.
