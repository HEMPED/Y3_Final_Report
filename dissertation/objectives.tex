\chapter{Project Objectives}
\label{ch:project_objectives}
This chapter outlines the milestones of the project. These objectives were derived from the background research and are designed to address the technical and theoretical challenges identified with traditional delegation systems.

To manage the project effectively, the objectives are divided into two categories: \textbf{core objectives}, which form the backbone of the implementation and are essential to meeting the project's main goals, and \textbf{extension objectives}, which provide additional insight or value. 

Later in the chapter, each objective is broken down into specific functional and non-functional requirements. This structure helps to clarify expectations, guide implementation, and provide clear criteria for evaluating whether each objective has been met.

Core Objectives:
\begin{enumerate}
    \item \textbf{Implement a Core Delegation Model into Vodle: } Build upon the existing, partially implemented delegation code within the vodle platform to create a fully functional system, including resolving key challenges such as cyclic delegations.

    \item \textbf{Implement Ranked Delegation into Vodle: } Add a backup delegation mechanism to vodle, allowing users to specify up to 3 delegates.

    \item \textbf{Implement a Weighte Delegation into Vodle: } Add functionality to vodle to delegate fractions of their rating to different delegates. Use the \textit{will come back to when research written up} system to calculate final ratings.

    \item \textbf{Implement Per-Option Delegation}: Allow users to delegate the ratings of specific options to different delegates.
\end{enumerate}

Extension Objective:
\begin{enumerate}
    \item \textbf{Simulate Delegation Mechanisms: } Perform agent-based modelling to analyse the effectiveness of various delegation systems, including those outlined in Objectives 1, 2 and 3.
\end{enumerate}

\section{Project Requirements}
%intro - talk about why we need to break down objectives further, define functional and non functional requirements.
The project objectives represent high-level goals that must be translated into specific, actionable requirements. This process is essential for clarifying the scope of the work, ensuring comprehensive coverage of each objective, and establishing a structured foundation for both implementation and evaluation.

To support this, requirements are organised into two categories: functional (F) and non-functional (NF). Functional requirements describe the core behaviours and features the system must support, while non-functional requirements define performance, usability, and other quality-related constraints \citep{sommervill_2016}. Distinguishing between these categories helps ensure that both the system's functionality and overall user experience are properly addressed.

Each requirement is formulated to be measurable and testable. This allows for objective evaluation during development, facilitates verification against the project goals, and helps identify areas for improvement as the system evolves.

\subsection{Implement a Core Delegation Model into Vodle}\label{subsec:requirements_core}
\subsubsection{Functional Requirements}
\begin{itemize}
    \item \textbf{FR1:} The system shall correctly handle the delegation process from invitation to acceptance.
    \begin{itemize}
        \item \textbf{FR1.1:} The system shall allow users to invite others to act as their delegate.
        \item \textbf{FR1.2:} The system shall allow invited users to accept delegation requests.
        \item \textbf{FR1.3:} The system shall prevent users from accepting their own delegation invitations.
        \item \textbf{FR1.4:} The system shall detect and prevent the formation of delegation cycles.
    \end{itemize}

    \item \textbf{FR2:} The system shall provide users with a clear view of their current delegation, including the ability to revoke it at any time.

    \item \textbf{FR3:} The system shall resolve delegations transitively, such that if User A delegates to B and B delegates to C, User C is the final casting voter for A's vote.

    \item \textbf{FR4:} The system shall allow users to override a delegate's decision for specific poll options by submitting a direct vote.
\end{itemize}

\subsubsection{Non-Functional Requirements}
\begin{itemize}
    \item \textbf{NFR1:} All delegation-related data must be stored in a JSON-encoded format, ensuring compatibility with the existing vodle CouchDB database.
    \item \textbf{NFR2:} Any changes to the database schema must be backward compatible, ensuring that existing data is not lost or corrupted during the upgrade process.
    \item \textbf{NFR3:} The system shall preserve user privacy by ensuring that individual voting preferences and delegation choices are not visible to other users. The only information visible to a delegated user shall be the final vote cast on their behalf.
    \item \textbf{NFR4:} The user interface for delegation shall be intuitive and accessible, with clear instructions and minimal friction to perform delegation actions.
\end{itemize}

\subsection{Implement Ranked Delegation into Vodle}\label{subsec:requirements_ranked_delegation}

\subsubsection{Functional Requirements}
\begin{itemize}
    \item \textbf{FR1:} The system shall allow users to specify a ranked list of up to 3 delegates for each poll, with the ranking applying to all options within that poll.
    
    \item \textbf{FR2:} The system shall apply the MinSum delegation rule to resolve each voter's delegation path based on their ranked list of delegates.
    
    \item \textbf{FR3:} The system shall allow users to override the ranked delegation by submitting a direct vote for specific poll options.
    
    \item \textbf{FR4:} The system shall provide users with a clear view of their ranked delegation choices, including the ability to alter their rankings or revoke them at any time.
\end{itemize}

\subsubsection{Non-Functional Requirements}
\begin{itemize}
    \item \textbf{NFR1:} The user interface for ranked delegation shall be intuitive and accessible, with clear instructions and minimal friction to perform delegation actions.
    
    \item \textbf{NFR2:} The system shall ensure that any data related to ranked delegation is stored in a JSON-encoded format, ensuring compatibility with the existing vodle CouchDB database.
\end{itemize}
\subsection{Implement Weighted Delegation into Vodle}\label{subsec:requirements_weighte}
\subsubsection{Functional Requirements}
\begin{itemize}
    \item \textbf{FR1:} The system shall allow users to delegate their vote to multiple delegates simultaneously for a single poll.

    \item \textbf{FR2:} The system shall allow users to assign a weight to each delegate such that the total weight does not exceed 0.99.

    \item \textbf{FR3:} The system shall use the algorithm described in Section~\ref{subsec:background_weighted_delegation} to calculate the final rating for each option based on the weights assigned to each delegate.
\end{itemize}

\subsubsection{Non-Functional Requirements}
\begin{itemize}
    \item \textbf{NFR1:} The system shall ensure that weighted delegation calculations are performed entirely on the client side to comply with vodle's CouchDB architecture.

    \item \textbf{NFR2:} All related data must be serialised as JSON strings to ensure compatibility with the CouchDB backend.

    \item \textbf{NFR3:} The user interface for weighted delegation shall be intuitive and allow users to adjust weights easily, using sliders or numeric inputs.
\end{itemize}
\subsection{Implement Per-Option Delegation}\label{subsec:requirements_option}
\subsubsection{Functional Requirements}
\begin{itemize}
    \item \textbf{FR1:} The system shall allow users to assign different delegates for each individual option or subset of options within a poll.

    \item \textbf{FR2:} The system shall ensure that each delegated option is resolved independently, using the appropriate delegate's vote for that option.

    \item \textbf{FR3:} The system shall allow users to override a delegate's vote for a specific option by submitting their own rating.

    \item \textbf{FR4:} The system shall provide a user interface for viewing or revoking each individual delegation.
\end{itemize}

\subsubsection{Non-Functional Requirements}
\begin{itemize}
    \item \textbf{NFR1:} The delegation interface must be intuitive and clearly indicate which delegate is assigned to each option, ensuring ease of use.

    \item \textbf{NFR2:} The delegation data must be serialised in a format compatible with CouchDB (e.g., JSON-encoded) to maintain compatibility with vodle's storage system.
\end{itemize}
\subsection{Simulate Delegation Mechanisms}\label{subsec:requirements_sim}
\subsubsection{Functional Requirements}
\begin{itemize}
    \item \textbf{FR1:} The simulation system shall model individual agents representing voters, each capable of voting, abstaining, or delegating their vote according to a selected delegation rule.

    \item \textbf{FR2:} The system shall support multiple delegation mechanisms, including standard transitive delegation, ranked delegation (with the MinSum delegation rule), and weighted delegation.

    \item \textbf{FR3:} The system shall allow configuration of simulation parameters such as number of agents, delegation probabilities and abstention rates.

    \item \textbf{FR4:} The system shall track and record key metrics such as vote concentration, number of super-voters, average chain length, vote loss due to abstentions or cycles, and decision quality.

    \item \textbf{FR5:} The system shall output simulation results in a structured format (e.g. CSV or JSON) for further analysis.
\end{itemize}

\subsubsection{Non-Functional Requirements}
\begin{itemize}
    \item \textbf{NFR1:} The simulation framework must be lightweight and easy to extend, enabling rapid experimentation with new delegation rules or metrics.

    \item \textbf{NFR2:} The system shall be developed using Mesa to take advantage of existing data science libraries such as NumPy, Pandas, and Matplotlib for analysis and visualisation.

    \item \textbf{NFR3:} The simulation design shall support reproducibility by enabling fixed random seeds and storing configuration settings alongside output data.

\end{itemize}

