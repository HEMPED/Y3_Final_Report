\chapter{Introduction}\label{ch:introduction}

\section{Context and Motivation}

In online communities, organisations, and even national governments, decision-making is a central pillar of collective action. Yet as participation scales, two traditional models of democracy struggle to meet the evolving needs of modern groups. In a \textbf{direct democracy}, individuals vote on issues themselves, ensuring transparency and personal agency. In contrast, \textbf{representative democracy} delegates this responsibility to elected individuals, offering scalability at the cost of responsiveness \citep{ford_delegative_2002, blum_liquid_2016}.

Each system introduces trade-offs. Direct democracy demands consistent, informed engagement from all participants -- an unrealistic expectation for most. Representative democracy, while more practical at scale, can result in power imbalances and diluted accountability between elections.

This project builds on vodle, a web-based decision-making platform which uses direct democracy. Users rate poll options on a scale from 0 to 100, with outcomes determined using a unique voting rule MaxParC. Vodle's design prioritises openness, flexibility, and fairness -- making it well-suited for participatory decisions where all members are encouraged to contribute.

However, in real-world settings, vodle's reliance on direct democracy reveals limitations. Many users lack the time, interest, or expertise to participate meaningfully in every decision. As decisions grow in number and complexity, expecting consistent, active engagement from all users becomes increasingly impractical. This challenge calls for a more adaptable system -- one that preserves transparency while accommodating varying levels of participation.

\subsection{Liquid Democracy}

Liquid democracy blends the strengths of direct and representative systems. It allows users to either vote directly or delegate their vote to someone they trust \citep{blum_liquid_2016}. Crucially, delegations are \textit{transitive}: if Alice delegates to Bob and Bob to Charlie, Charlie votes on behalf of all three. This flexible model lets users dynamically choose between direct participation and trusted representation, adapting to their interest and availability on a per-decision basis.

By enabling seamless movement between voting and delegation, liquid democracy offers a principled extension of vodle's model, allowing users to vary their participation over time while preserving system transparency and fairness. It enhances inclusiveness without sacrificing autonomy -- users can reclaim their vote at any time.

However, implementing liquid democracy introduces several challenges:
\begin{itemize}
  \item \textbf{Delegation cycles} -- votes can loop endlessly, preventing any vote from being cast.
  \item \textbf{Abstentions in delegation chains} -- a single abstaining delegate can nullify multiple dependent votes.
  \item \textbf{Disproportionate influence} -- some users may receive many delegations, becoming ``super-voters'' with excessive power.
\end{itemize}

Addressing these issues requires thoughtful technical solutions and intuitive interfaces that offer users visibility and control over their vote delegation.

\section{Project Aims}

This project aims to design and implement a robust, user-friendly liquid democracy system within vodle. By introducing features such as transitive, ranked, and per-option delegation, the project extends vodle's capabilities beyond direct voting -- enabling more inclusive, scalable, and context-sensitive decision-making.

\section{Structure of This Report}

The remainder of this report is structured as follows:

\begin{itemize}
  \item Chapter~\ref{ch:background} outlines the theoretical and technical background, including research into delegation models and related systems.
  \item Chapter~\ref{ch:project_objectives} defines the core and extension objectives of the project, along with detailed requirements.
  \item Chapter~\ref{ch:design_implementation} presents the architecture and implementation of the delegation features.
  \item Chapter~\ref{ch:evaluation} evaluates the system through testing and feedback.
  \item Chapter~\ref{ch:project_management} reflects on the planning, risks, and methodology used during development.
  \item Chapter~\ref{ch:conclusions} offers a summary of the project outcomes and directions for future work.
\end{itemize}
