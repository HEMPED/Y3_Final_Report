\chapter{Conclusion}\label{ch:conclusions}

This project has successfully designed, implemented, and evaluated a comprehensive liquid democracy system within the vodle platform. The work advances vodle from a system reliant on direct voting alone to one capable of supporting flexible, transitive, ranked, and weighted delegation, thereby greatly expanding user autonomy and participation options. Several complex delegation models were integrated -- including ranked delegation using the MinSum rule, per-option delegation, and trust-based weighted delegation -- with cycle prevention and client-side resolution mechanisms ensuring system consistency despite the architectural constraints of a CouchDB-backed, serverless design.

Following customer feedback (see Section~\ref{sec:feedback}), the weighted delegation model was selected as the final implementation for deployment. This variation captures nuanced trust relationships between users, aligns with vodle's emphasis on user autonomy and transparency, and ensures fine-grained control over vote distribution even under real-world participation fluctuations.

The project's technical contribution lies in the development of a client-side delegation framework compatible with real-time collaborative environments without requiring server-side computation. Special attention was given to resolving key challenges such as cycle detection, delegation graph resolution, and maintaining user privacy. Furthermore, this work proposes solutions for critical scalability issues, notably the potential database collisions caused by shared delegation maps, laying out pathways for future architectural improvements through strategies like per-user delegation documents.

This contribution is highly relevant to the field of Computer Science, particularly in areas such as computational social choice, participatory system design, and decentralised decision-making architectures. It demonstrates practical methods to balance decentralisation, flexibility, and reliability -- trade-offs that are central to many modern digital decision-making platforms.

Other developers and researchers can build upon this work by adopting or extending the delegation models presented here. The delegation structures and resolution algorithms are adaptable to a wide range of decision-making contexts beyond vodle, such as online governance platforms, collaborative filtering systems, and digital democracy initiatives.

The project represents a significant achievement in its integration of complex liquid democracy mechanisms within tight architectural constraints, preserving usability, efficiency, and trustworthiness for end users. It required extensive background research, careful technical design, and iterative validation to ensure correctness and scalability, all of which were successfully delivered within the project timeline.

Nonetheless, some limitations remain. The database collision issue, while identified and partially addressed, requires deeper architectural redesign for full resolution. Furthermore, the agent-based simulation component, though proposed and partially scoped, was not fully implemented within the available time. Finally, large-scale empirical validation on live user data remains an avenue for future work.