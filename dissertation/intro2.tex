\chapter{Introduction}\label{ch:introduction}

\section{Context and Motivation -- TODO}

In group settings ranging from online communities to organisations, collective decision making is both essential and difficult to scale. Direct democracy empowers individuals by letting everyone vote on every issue, but struggles with engagement as participation grows. Representative democracy improves scalability but often reduces accountability and flexibility \citep{ford_delegative_2002, blum_liquid_2016}

Liquid democracy is a hybrid model that aims to balance these trade-offs. It allows individuals to either vote directly or delegate their voting power to others they trust. This approach combines the transparency and agency of direct democracy with the scalability of representation, offering a dynamic alternative for participatory systems.

This project applies liquid democracy within vodle, a web-based platform for group decision making. Vodle allows users to express nuanced preferences across decision options and promotes open, participatory decision making. However, it inherits a common limitation: users may lack the time, interest, or confidence to engage with every decision, leading to underrepresentation and disengagement.

By integrating liquid democracy into vodle, the project aims to give users more flexibility in how their preferences are represented. Delegation features allow users to stay involved even when they choose not to vote directly, making the platform more scalable, inclusive, and responsive to varying levels of user engagement -- all while maintaining transparency and user control.

\section{Project Goals}

The goal of this project is to design and implement a liquid democracy system within vodle. This involves building flexible delegation mechanisms that support diverse participation styles and address the limitations of traditional direct voting.

The system introduces several key features:
\begin{itemize}
  \item \textbf{Ranked delegation}, allowing users to specify trusted delegates in order of preference.
  \item \textbf{Per-option delegation}, enabling different delegates for different decision topics.
  \item \textbf{Vote splitting}, using a trust-based model to distribute voting power across multiple delegates.
\end{itemize}

These features are designed to enhance participation, reduce the concentration of voting power, and improve the resilience of the platform. The project focuses on creating an intuitive and efficient implementation that aligns with vodle's existing architecture and emphasises user autonomy.

\section{Structure of This Report}

The remainder of this report is structured as follows:

\begin{itemize}
  \item Chapter~\ref{ch:background} outlines the theoretical and technical background, including research into delegation models and related systems.
  \item Chapter~\ref{ch:project_objectives} defines the core and extension objectives of the project, along with detailed requirements.
  \item Chapter~\ref{ch:design_implementation} presents the architecture and implementation of the delegation features.
  \item Chapter~\ref{ch:evaluation} evaluates the system through testing and feedback.
  \item Chapter~\ref{ch:project_management} reflects on the planning, risks, and methodology used during development.
  \item Chapter~\ref{ch:conclusions} offers a summary of the project outcomes and directions for future work.
\end{itemize}